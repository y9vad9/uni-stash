\chapter{Calculations in idempotent algebras}


\section {Tropical algebras}
You can work in the following tropical algebras :

SEMIFIELDS\\
1) On the set of integers $ {\ mathbb Z} $ we define: \\
$ZMaxPlus$,  
$ZMinPlus$.\\
2) On the set of numbers ${\mathbb R}$ we define:\\
$RMaxPlus$, 
$RMinPlus$,  
$RMaxMult$,  
$RMinMult$.\\
3) On the set of numbers ${\mathbb R}64$ we define:\\
$R64MaxPlus$, 
$R64MinPlus$,  
$R64MaxMult$,  
$R64MinMult$.\\

SEMIRINGS\\
1) On the set of numbers ${\mathbb Z}$ we define:\\ 
$ZMaxMin$,
$ZMinMax$,
$ZMaxMult$,
$ZMinMult$.\\
2) On the set of numbers ${\mathbb R}$ we define:\\
$RMaxMin$, 
$RMinMax$.\\
3) On the set of numbers ${\mathbb R}64$ we define:\\
$R64MaxMin$, 
$R64MinMax$.

 

Examples of tropical algebras: 

SPACE = ZMaxPlus [x,  y,  z]; 

SPACE = R64MinMult [u,  v];  

SPACE = RMaxMin [u,  v]. 

 An example of a simple problem in a semiring
 $ZMaxMult$.
%begindelete
\smallskip

\underline{Example 1. }

%\vspace*{-2mm}
%enddelete
\begin{verbatim}
SPACE = ZMaxMult[x, y];
a = 2; b = 9; c = a + b; d = a*b; \print(c, d)
\end{verbatim}
%begindelete

Returns:\\
$c = 9; $\\
$d = 18.$
%enddelete

In the remaining sections of this chapter we have given some examples of problems that are solved in the tropical algebra, which are semi-fields.
%begindelete
\section{Solving systems of linear algebraic equations}
The command $\backslash solveLAETropic(A, b)$ enables us to find a particular solution of the system 
$Ax = b$.
%begindelete
\smallskip

\underline{Example 2. }

%\vspace*{-2mm}
%enddelete
\begin{verbatim}
SPACE = R64MaxPlus[x, y];
A = [
  [1.00, 1.00, 0.00],
  [2.00, 0.00, 3.00],
  [3.00, 4.00, 2.00]
];
b = [8.00, 7.00, 11.00];
X = \solveLAETropic(A, b); 
\print(X);
\end{verbatim}
%begindelete

Returns:\\
$X = \left(\begin{array}{c}
5.00\\
7.00\\
4.00\\
\end{array}\right)
$ 
%enddelete
 
\section{Solving systems of linear algebraic inequalities}
The command $\backslash solveLAITropic(A,b)$ enables us to find a particular solution of the system of inequalities 
%begindelete
\smallskip

\underline{Example 3. }

\vspace*{-3mm}
%enddelete
\begin{verbatim}
SPACE = R64MaxPlus[x, y];
A = [
  [1.00, 1.00, 0.00],
  [2.00, 0.00, 3.00],
  [3.00, 4.00, 2.00]
];
b = [10.00, 7.00, 11.00]; 
X = \solveLAITropic(A, b); 
\print(X);
\end{verbatim}
%begindelete

Returns:\\
$X=[(-\infty,5.00],(-\infty,7.00],(-\infty,4.00]]$ 
%enddelete

%begindelete
\smallskip

\underline{Example 4. }

\vspace*{-3mm}
%enddelete
\begin{verbatim}
SPACE = ZMinPlus[x, y];
A = [
  [1, 1, 0],
  [2, 0, 3],
  [3, 4, 2]
];
b = [10, 7, 11];
X = \solveLAITropic(A, b); 
\print(X);
\end{verbatim}
%begindelete

Returns:\\
$X=[[9,\infty),[9,\infty),[10,\infty)]$ 
%enddelete

\section{The solution of the Bellman equation}
\subsection {The homogeneous  Bellman equation}
The command $\backslash BellmanEquation(A)$ enables us to find a solution of the homogeneous  Bellman equation
  $Ax = x$.
%begindelete
\smallskip

\underline{Example 5.}

\vspace*{-3mm}
%enddelete
\begin{verbatim}
SPACE = R64MaxPlus[x, y];
A = [
  [0.00, -2.00, -\infty, -\infty],
  [-\infty, 0.00, 3.00, -1.00],
  [-1.00, -\infty, 0.00, -4.00],
  [2.00, -\infty, -\infty, 0.00]
]; 
X = \BellmanEquation(A); 
\print(X);
\end{verbatim}
%begindelete

Returns:\\
$$
X=\left(\begin{array}{cccc}
0.00 & -2.00 & 1.00 & -3.00\\
2.00 & 0.00 & 3.00 & -1.00\\
-1.00 & -3.00 & 0.00 & -4.00\\
2.00 & 0.00 & 3.00 & 0.00
\end{array}\right) \left(\begin{array}{c}
v_{1}\\
v_{2}\\
v_{3}\\
v_{4}
\end{array}\right), \forall v_{1}, v_{2}, v_{3}, v_{4}.$$
%enddelete
\subsection {The inhomogeneous  Bellman equation}
The command $\backslash BellmanEquation(A,b)$ enables us to find a solution of the inhomogeneous  Bellman equation $Ax\oplus b=x$.
%begindelete
\smallskip

\underline{Example 6. }

\vspace*{-3mm}
%enddelete
\begin{verbatim}
SPACE = R64MaxPlus[x, y];
A = [
  [0.00, -2.00, -\infty, -\infty],
  [-\infty, 0.00, 3.00, -1.00],
  [-1.00, -\infty, 0.00, -4.00],
  [2.00, -\infty, -\infty, 0.00]
];
b = [[1], [-\infty], [-\infty], [-\infty]]; 
X = \BellmanEquation(A, b); 
\print(X);
\end{verbatim}
%begindelete

Returns:\\
$$X=
 \left(\begin{array}{cccc}
0.00 & -2.00 & 1.00 & -3.00\\
2.00 & 0.00 & 3.00 & -1.00\\
-1.00 & -3.00 & 0.00 & -4.00\\
2.00 & 0.00 & 3.00 & 0.00
\end{array}\right)
\left(\begin{array}{c}
v_{1}\\
v_{2}\\
v_{3}\\
v_{4}
\end{array}\right)
\oplus\left(\begin{array}{c}
1.00\\
3.00\\
0.00\\
3.00
\end{array}\right),$$\\ $$\forall v_{1}, v_{2}, v_{3}, v_{4}.$$
%enddelete
\section{The solution Bellman inequality}
\subsection{The homogeneous Bellman inequality}
 
The command $\backslash BellmanInequality(A)$ enables us to find a solution of the homogeneous Bellman inequality $Ax\leq x$.

\subsection{The inhomogeneous Bellman inequality}
The command $\backslash BellmanInequality(A, b)$ enables us to find a solution of the inhomogeneous Bellman inequality $Ax\oplus b\leq x$.

\section{Finding the shortest path between the vertices of the graph}
\subsection {Calculation of the table of shortest distances for all vertices of the graph}
Let $A=(x_{ij})$ be matrix of distances between adjacent vertices. We put $x_{ii}$=0 $\forall i$ and we put $x_{ij}=\infty$, if there is no edge connecting vertices i and j.
The command $\backslash searchLeastDistances(A)$ allows you to find the smallest distance between all the nodes of the graph.
This results in a matrix of shortest paths between all vertices.
 
%begindelete
\smallskip

\underline{Example 7. }

\vspace*{-3mm}
%enddelete
\begin{verbatim}
SPACE = R64MinPlus[x, y];
A = [
  [0.00, 7.00, 9.00, \infty, \infty, 14.00],
  [7.00, 0.00, 10.00, 15.00, \infty, \infty],
  [9.00, 10.00, 0.00, 11.00, \infty, 2.00],
  [\infty, 15.00, 11.00, 0.00, 6.00, \infty],
  [\infty, \infty, \infty, 6.00, 0.00, 9.00],
  [14.00, \infty, 2.00, \infty, 9.00, 0.00]
];
B = \searchLeastDistances(A);
\print(B);
\end{verbatim}
%begindelete

Returns:\\
$$B= \left(\begin{array}{cccccc}
0.00 & 7.00 & 9.00 & 20.00 & 20.00 & 11.00\\
7.00 & 0.00 & 10.00 & 15.00 & 21.00 & 12.00\\
9.00 & 10.00 & 0.00 & 11.00 & 11.00 & 2.00\\
20.00 & 15.00 & 11.00 & 0.00 & 6.00 & 13.00\\
20.00 & 21.00 & 11.00 & 6.00 & 0.00 & 9.00\\
11.00 & 12.00 & 2.00 & 13.00 & 9.00 & 0.00
\end{array}\right) $$
%enddelete
\subsection {Calculation of the shortest distances between two vertices of the graph}
Let $A=(x_{ij})$ be matrix of distances between adjacent vertices. We put $x_{ii}$=0 $\forall i$ and we put $x_{ij}=\infty$, if there is no edge connecting vertices $i$ and $j$.

The command $\backslash findTheShortestPath(A, i, j)$ allows you to find the shortest path between nodes $i$ and $j$.
 %begindelete
\smallskip

\underline{Example 8. }

\vspace*{-3mm}
%enddelete
\begin{verbatim}
SPACE = R64MinPlus[x, y];
A = [
  [0.00, 7.00, 9.00, \infty, \infty, 14.00],
  [7.00, 0.00, 10.00, 15.00, \infty, \infty],
  [9.00, 10.00, 0.00, 11.00, \infty, 2.00],
  [\infty, 15.00, 11.00, 0.00, 6.00, \infty],
  [\infty, \infty, \infty, 6.00, 0.00, 9.00],
  [14.00, \infty, 2.00, \infty, 9.00, 0.00]
];
X = \findTheShortestPath(A, 0, 4);
\print(X);
\end{verbatim}
%begindelete

Returns:\\
$X=[[0, 2, 5, 4]]$
%enddelete
 
