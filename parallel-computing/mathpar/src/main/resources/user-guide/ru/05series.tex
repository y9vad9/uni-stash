\chapter{ Ряды}

Ряд задается в виде f=$\backslash {\mathbf sum}$ \_\{i=k\} $\widehat{\ }{}$\{ $\backslash$infty\} $F(i, x, y,\ldots, z)$,  где $i$~--- индекс суммирования,  $k$~--- начальное значение $i$,  $F(i, x, y,\ldots, z)$~--- функция многих переменных, т.е.  $F$ может зависеть и от $i$. 

Над рядами определены такие арифметические операции как сложение, вычитание и умножение.

Пусть $f$ и $g$~--- ряды.  

Для сложения двух рядов необходимо выполнить команду 
 \comm{seriesAdd}{(f, g)}. 

Для разности двух рядов необходимо выполнить команду
\comm{seriesSubtract}{(f, g)}. 

Для умножения двух рядов необходимо выполнить команду
 \comm{seriesMultiply}{(f, g)}. 

Для разложения функции в ряд Тейлора с определенным
количеством членов ряда необходимо выполнить команду \comm{teilor}{(f, point, num)},  где
$f$~--- функция,  $point$~--- точка,  $num$~--- общее количество членов ряда. 

\underline{Примеры. }

\vspace*{-2mm}
\begin{verbatim}
SPACE=R[x, y];
f=\sum_{i=2}^{\infty} (2x^i y b i);
g=\sum_{i=4}^{\infty} (x^i a\sin(a i x));
h=\seriesAdd(f, g); 
\print(f, g, h);
\end{verbatim}
%begindelete

Результат выполнения:\\
$$f = \sum_{i=2}^{\infty} 2(x)^iybi;$$
$$g = \sum_{i=4}^{\infty} (x)^ia\sin(aix);$$
$$h = \sum_{i=4}^{\infty} (2(x)^iybi+(x)^ia\sin(aix))+\sum_{i=2}^{3}(2x^iybi);$$

%enddelete
\begin{verbatim}
SPACE=R[x, y];
f=\sum_{i=1}^{\infty} (x^i y i\cos(b));
g=\sum_{i=2}^{\infty} (5x^i a\cos(axi));
h=\seriesSubtract(f, g); 
\print(f, g, h);
\end{verbatim}
%begindelete

Результат выполнения:\\
$$f = \sum_{i=1}^{\infty} (x)^iyi\cos(b);$$
$$g = \sum_{i=2}^{\infty} 5(x)^ia\cos(axi);$$
$$h = \sum_{i=2}^{\infty} ((x)^iyi\cos(b)-5(x)^ia\cos(axi))+xy\cos(b);$$
%enddelete
\begin{verbatim}
SPACE=R[x, y];
f=\sum_{i=0}^{\infty} (2x^i y b i);
g=\sum_{i=2}^{\infty} (5y^i x^2 b i\cos(a_1 x));
h=\seriesSubtract(f, g); 
\print(f, g, h);
\end{verbatim}
%begindelete

Результат выполнения:\\
$$f = \sum_{i=0}^{\infty} 2(x)^iybi;$$
$$g = \sum_{i=2}^{\infty} 5y^ix^2bi\cos(a_1x);$$
$$h = \sum_{i=2}^{\infty} (2x^iybi-5y^ix^2bi\cos(a_1*x))+\sum_{i=0}^{1}(2x^iybi);$$

%enddelete
\begin{verbatim}
SPACE = R[x, y];
f = \sum_{a = 6}^{\infty} (x a a_0);
g = \sum_{a = 9}^{\infty} (56x^4 a \cos(a_1 x));
h = \seriesMultiply(f, g); 
\print(f, g, h);
\end{verbatim} 
%begindelete

Результат выполнения:\\
$$f = \sum_{a=6}^{\infty} xaa_0;$$
$$g = \sum_{a=9}^{\infty} 56. 00x^4a\cos(a_1x);$$
$$h = \sum_{a_2=6}^{\infty} \sum_{a=9}^{\infty} xa_2a_0\cdot 56. 00x^4a\cos(a_1x);$$
%enddelete

\begin{verbatim}
SPACE=R[x, y];
f = \sum_{a = 6}^{\infty} (\sin(xa)\cos(g y) a_0);
g = \sum_{a = 9}^{\infty} (6x^2 a \sin(a x y^2));
h = \seriesMultiply(f, g); 
\print(f, g, h);
\end{verbatim}
%begindelete

Результат выполнения:\\
$$f = \sum_{a=6}^{\infty} \sin(xa)\cos(\sum_{a=9}^{\infty} 56. 00x^4a\cos(a_1x)y)a_0;$$
$$g = \sum_{a=9}^{\infty} 6. 00x^2a\sin(axy^2);$$
$$h = \sum_{a_2=6}^{\infty} \sum_{a=9}^{\infty} \sin(xa_2)\cos(\sum_{a_2=9}^{\infty} 56. 00x^4a_2\cos(a_1x)y)a_06. 00x^2a\sin(axy^2);$$
%enddelete

\begin{verbatim}
SPACE = R[x]; 
FLOATPOS = 15;
a = \teilor(\sin(x), 0, 7);
c = \value(a); 
\print(a, c);
\end{verbatim} 
%begindelete
Результат выполнения:\\
$a = ((-x^{7})/(7\mathbf{}!)+x^{5}/(5\mathbf{}!)+(-x^{3})/(3\mathbf{}!)+x/(1\mathbf{}!)); $\\
$c = (-0. 000198412698412x^{7}+0. 008333333333333x^{5}-0. 166666666666666x^{3}+x)$.  

\section{Контрольные задания}
В  Mathpar 
\begin{itemize}
 \item разложите функцию $f(x) = \sin^2(5x - 1)$ в ряд Тейлора, 
 \item создайте два случайных ряда с полиномиальными членами,  используя генерацию полиномов.  Найдите сумму,  разность и произведение полученных рядов. 
  \end{itemize}
%enddelete
